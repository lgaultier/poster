\documentclass[portrait,a0,final,a4resizeable]{a0poster}
%\documentclass[portrait,a0b,final]{a0poster}
%%% Option "a4resizeable" enables the poster to be resizable using the command: psresize -pa4 poster.ps poster-a4.ps
%   with landscape option: psresize -W800 -H600 poster.ps poter-a4.ps
%   and replace in "poster-a4.ps" the line:  
%   %%BoundingBox: 0 0 2594 3402  with  %%BoundingBox: 0 0 600 800 
%   For final printing, please remove option "a4resizeable" !!
\usepackage{epsfig}
\usepackage{multicol}
\usepackage{natbib}
\usepackage{pstricks,pst-grad}
\usepackage{graphicx}
%\usepackage{mathptmx}
%\usepackage{mathpazo}
\usepackage{amsbsy,amsmath}
\usepackage{tikz} %add schemes
\usetikzlibrary{shapes} %add diamonds shape to schemes
\usepackage{pifont} %to use arrows
\usepackage{array} %vertically centered cells in tabular
%\usepackage{verbatim} % to enable comments
%%%%%%%%%%%%%%%%%%%%%%%%%%%%%%%%%%%%%%%%%%%%
% Definition of some variables and colors %%
%%%%%%%%%%%%%%%%%%%%%%%%%%%%%%%%%%%%%%%%%%%%

\newcommand{\unnumberedcaption}%

\newcommand{\BIBand}{and}

\newcommand{\Deg}{$^{\circ} ~$}
%\DeclareTextSymbol{\deg}{T1}{6}
%\DeclareTextSymbol{\degre}{OT1}{23}

\setlength{\bibsep}{0.0pt}
\def\refname{\normalsize \bf References}
%\setlength{\columnsep}{3cm}
%\setlength{\columnseprule}{2mm}
%\setlength{\parindent}{0.0cm}

%\renewcommand{\vec}{\underline}
%\renewcommand{\rho}{\varrho}
%\renewcommand{\phi}{\varphi}
\setlength{\columnsep}{0.1cm}
\setlength{\columnseprule}{0mm}
\setlength{\parindent}{0.0cm}

%%%%%%%%%%%%%%%%%%%%%%%%%%%%%%%%%%%%%%%%%%%%%%%%%%%%
%%%               Background                     %%%
%%%%%%%%%%%%%%%%%%%%%%%%%%%%%%%%%%%%%%%%%%%%%%%%%%%%

\newcommand{\background}[3]{
  \newrgbcolor{cgradbegin}{#1}
  \newrgbcolor{cgradend}{#2}
  \psframe[fillstyle=gradient,gradend=cgradend,
  gradbegin=cgradbegin,gradmidpoint=#3](0.,0.)(1.\textwidth,-1.\textheight)
}


%%%%%%%%%%%%%%%%%%%%%%%%%%%%%%%%%%%%%%%%%%%%%%%%%%%%
%%%                Poster                        %%%
%%%%%%%%%%%%%%%%%%%%%%%%%%%%%%%%%%%%%%%%%%%%%%%%%%%%

\newenvironment{poster}{
  \begin{center}
  \begin{minipage}[c]{1.0\textwidth}
}{
  \end{minipage} 
  \end{center}
}


%%%%%%%%%%%%%%%%%%%%%%%%%%%%%%%%%%%%%%%%%%%%%%%%%%%%
%%%                pcolumn                       %%%
%%%%%%%%%%%%%%%%%%%%%%%%%%%%%%%%%%%%%%%%%%%%%%%%%%%%

%  \flushcolumns %top and bottom baselines of all columns align.
\newenvironment{pcolumn}[1]{
  \begin{minipage}{#1\textwidth}
%  \begin{center}
}{
%  \end{center}
  \end{minipage}
}

%%%%%%%%%%%%%%%%%%%%%%%%%%%%%%%%%%%%%%%%%%%%%%%%%%%%
%%%                pboxx                         %%%
%%%%%%%%%%%%%%%%%%%%%%%%%%%%%%%%%%%%%%%%%%%%%%%%%%%%
\newrgbcolor{lcolor}{0. 0. 0.80}
\newrgbcolor{gcolor1}{1. 1. 1.}
\newrgbcolor{gcolor2}{.80 .80 1.}

\newcommand{\pboxx}[4]{
\begin{minipage}[t][#2][t]{#1}
#4
\end{minipage}
}

%%%%%%%%%%%%%%%%%%%%%%%%%%%%%%%%%%%%%%%%%%%%%%%%%%%%
%%%                pbox                          %%%
%%%%%%%%%%%%%%%%%%%%%%%%%%%%%%%%%%%%%%%%%%%%%%%%%%%%
\newrgbcolor{lcolor}{0. 0. 0.80}
\newrgbcolor{gcolor1}{1. 1. 1.}
\newrgbcolor{gcolor2}{.80 .80 1.}

\newcommand{\pbox}[4]{
         \psshadowbox[#3]{
            \begin{minipage}[t][#2][t]{#1} #4
            \end{minipage}
         }
}


%%%%%%%%%%%%%%%%%%%%%%%%%%%%%%%%%%%%%%%%%%%%%%%%%%%%
%%%                pfig                          %%%
%%%%%%%%%%%%%%%%%%%%%%%%%%%%%%%%%%%%%%%%%%%%%%%%%%%%
% \pfig - replacement for \figure
% necessary, since in multicol-environment
% \figure won't work

\newcommand{\pfig}[3][0]{
\begin{center}
  \vspace{1cm}
  \includegraphics[width=#3\hsize,angle=#1]{#2}
  \nobreak\medskip
\end{center}}


%%%%%%%%%%%%%%%%%%%%%%%%%%%%%%%%%%%%%%%%%%%%%%%%%%%%
%%%                pcaption                      %%%
%%%%%%%%%%%%%%%%%%%%%%%%%%%%%%%%%%%%%%%%%%%%%%%%%%%%
% \pcaption - replacement for \caption
% necessary, since in multicol-environment \figure and
% therefore \caption won't work

%\newcounter{figure}
\setcounter{figure}{1}
\newcommand{\pcaption}[1]{
  \vspace{0.cm}
  \begin{quote}
    {{\sc Figure} \arabic{figure}: #1}
  \end{quote}
  \vspace{1cm}
  \stepcounter{figure}
}
%%%%%%%%%%%%%%%%%%%%%%%%%%%%%%%%%%%%%%%%%%%%%%%%%%%%
%%%                myfig                         %%%
%%%%%%%%%%%%%%%%%%%%%%%%%%%%%%%%%%%%%%%%%%%%%%%%%%%%
% \myfig - replacement for \figure
% necessary, since in multicol-environment 
% \figure won't work

\newcommand{\myfig}[3][0]{
\begin{center}
  \vspace{0.10cm}
  \includegraphics[width=#3\hsize,angle=#1]{#2}
%  \nobreak\medskip
\end{center}}


%%%%%%%%%%%%%%%%%%%%%%%%%%%%%%%%%%%%%%%%%%%%%%%%%%%%
%%%                mycaption                     %%%
%%%%%%%%%%%%%%%%%%%%%%%%%%%%%%%%%%%%%%%%%%%%%%%%%%%%
% \mycaption - replacement for \caption
% necessary, since in multicol-environment \figure and
% therefore \caption won't work

%\newcounter{figure}
\setcounter{figure}{1}
\newcommand{\mycaption}[1]{
  \vspace{0.0cm}
  \begin{quote}
    {\small{#1}}
  \end{quote}
  \vspace{0cm}
  \stepcounter{figure}
}


% MES DEFINITIONS DES COULEURS:
\newrgbcolor{lcolor}{0. 0. 0.80}
\newrgbcolor{gcolor1}{1. 1. 1.}
\newrgbcolor{gcolor2}{.80 .80 1.}
\newrgbcolor{gradbegin}{0 0 0.8}
\newrgbcolor{gradend}{1 1 1}
\newrgbcolor{lightblue}{0. 0. 0.80}
\newrgbcolor{lightyellow}{1 1 0.}
\newrgbcolor{customcolor}{1 0.75 0.}
\newrgbcolor{customcolor2}{1 0.5 0.}
\newrgbcolor{customcolor3}{0 0.8 0.2}
\newrgbcolor{white}{1. 1. 1.}
\newrgbcolor{whiteblue}{.80 .80 1.}
\newrgbcolor{mycyan}{.20 .60 .60}

%%%%%%%%%%%%%%%%%%%%%%%%%%%%%%%%%%%%%%%%%%%%%%%%%%%%%%%%%%%%%%%%%%%%%%
%%% Begin of Document
%%%%%%%%%%%%%%%%%%%%%%%%%%%%%%%%%%%%%%%%%%%%%%%%%%%%%%%%%%%%%%%%%%%%%%

\begin{document}
\background{0.2 0.6 0.6}{1. 1. 1.}{0.5}

\vspace*{2cm}

%\newrgbcolor{lightblue}{0. 0. 0.80}
%\newrgbcolor{white}{1. 1. 1.}
%\newrgbcolor{whiteblue}{.80 .80 1.}

%\newrgbcolor{darkred}{0.6 0 0}
%\newrgbcolor{lightblue2}{0.2 0.2 1}
%\newrgbcolor{lightblue}{0.5 0.5 1}
%\newrgbcolor{white}{1. 1. 1.}
%\newrgbcolor{whiteblue}{.80 .80 1.}
%\newrgbcolor{mycyan}{rgb}{0.2,0.6,0.6}



\begin{poster}
%%%%%%%%%%%%%%%%%%%%%
%%% Header
%%%%%%%%%%%%%%%%%%%%%
\begin{center}
\begin{pcolumn}{0.98}
  \begin{center}
  \pbox{0.95\textwidth}{}{linewidth=2mm,framearc=0.3,linecolor=mycyan,shadowcolor=white,fillstyle=gradient,gradangle=0,gradbegin=white,gradend=white,gradmidpoint=1.0,framesep=1em}{
 
%%% Logo
  \begin{minipage}[l][9cm][c]{0.1\textwidth}
  \begin{center}
    \includegraphics[width=0.80\textwidth,angle=0]{./pict/logo/logo_cnrs.eps}
    \includegraphics[width=1.19\textwidth,angle=0]{./pict/logo/logo_cnes2.eps}
  \end{center}
  \end{minipage}
%%% Titre
  \begin{minipage}[c][9cm][c]{0.78\textwidth}
  \begin{center}
    {\sc \huge \bf On the Joint Use of High Resolution Tracer Images and Altimetric Data for the Control of Ocean Circulations}\\[10mm]
    {\Large Lucile Gaultier$^{1}$, Jacques Verron$^{1}$, Jean-Michel Brankart$^{1}$ and Pierre Brasseur$^{1}$} \\[5mm]
    {\large {$^{1}$ \it Laboratoire des Ecoulements G\'eophysiques et Industriels, UMR 5519, CNRS, Grenoble, France} %\\
%     %{$^{2}$ \it Departement of Oceanography, The Florida State University, Tallahassee, USA}\\
    } 
  \end{center}
  \end{minipage}
%%%Logo
  \begin{minipage}[r][9cm][c]{0.1\textwidth}
  \begin{center}
    \includegraphics[width=0.8\textwidth,angle=0]{./pict/logo/logo_meom.eps}
    \includegraphics[width=0.9\textwidth,angle=0]{./pict/logo/logo_legi.eps}
  \end{center}
  \end{minipage} 
  }  %pbox
  \end{center}
\end{pcolumn}
\end{center}
 
 
 
%%% Begin of Multicols-Environment
\begin{multicols}{2}
 
  \newcommand{\etal}{{\it et al.}}
  \newcommand{\DegN}{$^{\circ}$N~}
  \newcommand{\DegW}{$^{\circ}$W~}
  \newcommand{\DegE}{$^{\circ}$E~}
  \newcommand{\DegS}{$^{\circ}$S~}
  
% %%%%%%%%%%%%%%%%%%%%%%%%%%%%%%%%%%%%%%%%%%%%%%%%%%%%%%%%%%%%%%%%%%%%%%%%%%%%%%%%%%%%%%%%%%%%%%%%%%%%%%%%%%%%%%%%%%%%%%%%%%%%%%%%%%%%%%%%%%%%
% %%%% Introduction
  \begin{center}
% \begin{pcolumn}{0.32}
  \pbox{0.92\columnwidth}{30cm}{linewidth=2mm,framearc=0.1,linecolor=mycyan, fillstyle=gradient,gradangle=0,gradbegin=white,gradend=white, gradmidpoint=1.0,framesep=1em}{
    \begin{center} 
      { \bf \Large CONTEXT \\[5mm]}%MOTIVATIONS 
    \end{center}
    {\large
      \ding{226} Lack of resolution in time and space for altimetrice Data $\rightarrow$ Observation of the submesoscale resolution ($\simeq$ 50~km) is not possible \\
      \ding{226} Submesoscale filaments can be observed using tracer sensors (SST or Ocean Color image, resolution of image as low as 200 m and a map a day)
      \begin{center}
      \pbox{0.90\columnwidth}{}{linewidth=0.5mm,framearc=0.2,linecolor=black, fillstyle=gradient,gradangle=0,gradbegin=white,gradend=mycyan, gradmidpoint=0.0,framesep=3mm, shadowsize=0pt}{ \bf How to correct altimetric mesoscale velocity using sub-mesoscale tracer observations from space? }
      \end{center}
      \ding{226} Some studies brought to light the connection between mesoscale velocities and tracer patterns \citep{Dovidio2004, Lehahn2007}, but correcting mesoscale velocities using tracer images has never been done before.
%%%   \begin{center}
      \begin{tikzpicture}
%%%    \node[color=blue, text width=10cm, text centered] (UV) at (16,6.8) {Mesoscale field};
        \node[color=black, text width=09.73cm, text centered] (pUV) at (16,0.48) {
        \includegraphics[width=1\linewidth]{pict/aviso_20079_tas.eps}};
        \node[color=black, text width=20cm, text centered] (cUV) at (17,-6.) {\small{Velocity map, Tasmania region, December 22, 2004}};
%%%        \mycaption{Velocity map, Tasmania region, December 22, 2004}
%%%         };
%%    \node[color=green, text width=15cm, text centered] (TRA) at (0,6.7) {Sub-mesoscale tracer image};
       \node[color=black, text width=10cm, text centered] (pTRA) at (0,0) {
       \includegraphics[width=1\linewidth]{pict/A2004358041000_L2_LAC_OC.eps}};
        \node[color=black, text width=20cm, text centered] (cUV) at (0,-6.) {\small{Chlorophyll, Tasmania region, December 22, 2004}};
%        \mycaption{Chlorophyll, Tasmania region, December 22, 2004}
 %      };
        \node[draw] (int) at (8,2.1) {\large{?}};
%  \draw[arrows={triangle 45-triangle 45}] (pTRA)--(16,0);
%  \draw[->, thick,double distance=4pt , >=stealth] (pTRA)--(10,0);
        \draw[->, ultra thick, >=latex] (pTRA)--(10,0);
      \end{tikzpicture}\\ 
      \ding{226} 
       Use Finite-Size Lyapunov Exponents (FSLE) as a go-between variable.

    } %large
  } %pbox
  \end{center}
 
  \vspace*{0.1cm}

%%%%%%%%%%%%%%%%%%%%%%%%%%%%%%%%%%%%%%%%%%%%%%%%%%%%%%%%%%%%%%%%%%%%%%%%%%%%%%%%%%%%%%%%%%%%%%%%%%%%%%%%%%%%%%%%%%%%%%%%%%%%%%%%%%%%%%%%%%
\begin{center}
  \pbox{0.90\columnwidth}{58cm}{linewidth=2mm, framearc=0.1, linecolor=mycyan, fillstyle=gradient, gradangle=0, gradbegin=white, gradend=white, gradmidpoint=1.0, framesep=1em}{
    \begin{center}
      {\sc \bf \Large DATA IMAGE ASSIMILATION\\[5mm]}
    \end{center}
    \vspace*{0.2cm}
    {\large

\iffalse
    \node[color=black, text width=10.73cm, text centered] (ptra) at (10,5.48) {
    \includegraphics[width=1\linewidth]{./pict/s_atl/chl2006019.eps}};
    \node[color=black, text width=08.73cm, text centered] (ptra) at (20,5.48) {
    \includegraphics[width=1\linewidth]{./pict/s_atl/sst_20471_small.eps}};
    \node[color=black, text width=08.00cm, text centered] (puv) at (30,5.48) {
    \includegraphics[width=1\linewidth]{./pict/s_atl/aviso_20471_s_atl2_oc.eps}};
    \node[color=black, text width=08.73cm, text centered] (ph) at (40,5.48) {
    \includegraphics[width=1\linewidth]{./pict/s_atl/aviso_h_020471_bin0.eps}};
    \draw[->, very thick, >=stealth] (ph)--(puv) ;

    \node[color=black, text width=08.73cm, text centered] (ptra) at (10,-10) {
    \includegraphics[width=1\linewidth]{./pict/s_atl/sst_20471_small.eps}};
    \node[color=black, text width=08.73cm, text centered] (puv) at (16,-10) {
    \includegraphics[width=1\linewidth]{./pict/s_atl/fsle_48_stat_reg_20471__bin0.eps}};


    \node[draw,color=blue, text width=10cm, text centered] (J) at (16,-15) {Cost function};
    \node[color=black, text width=10cm, text centered] (fJ) at (16,-17) {$$J=\| I_{obs} - I_{FSLE}(u)\|$$ +background term};
\fi
    \begin{tikzpicture}
    \tikzset{pname/.style={draw,rectangle,rounded corners=6pt},text centered,color=black}
    \tikzstyle{fleche}=[->,>=latex,line width=1.6mm]
    \node[pname,text width=7cm] (OBS) at (9,3.98) {Observations};
    \node[pname,text width=7cm] (IM) at (9,-6.08) {Image};
    \draw[fleche] (OBS)--(IM);
    \draw[dashed,line width=0.5mm] (13.75,6.98)--(13.75,-7.1);

    \node[pname,text width=10cm] (TRA) at (20.5,6.48) {Sub-mesoscale tracer};
    \node[pname,text width=10cm] (DYN) at (32.5,6.48) {Dynamics};
    \draw[dashed,line width=0.5mm] (26.5,6.98)--(26.5,-7.1);
    \node[pname,text width=10cm] (OBSTRA) at (20.5,2.08) {MODIS observation of Sea Surface Temperature and/or Chlorophylle};
    \node[pname,text width=10cm] (OBSSSH) at (32.5,3.38) {Sea Surface Height from Altimetry};
    \node[pname,text width=10cm] (OBSUV) at (32.5,-1.08) {Corresponding geostrophic velocity $\vec{u}$};
    \draw[fleche] (OBSSSH)--(OBSUV);

    \node[pname, text width=10cm] (IMTRA) at (20.5,-5.88) {Image of tracer filaments $I_{tracer}$};
    \node[pname,text width=10cm] (IMDYN) at (32.5,-5.88) {Image of FSLE $I_{FSLE}(\vec{u})$};
    \draw[fleche] (OBSTRA)--(IMTRA);
    \draw[fleche] (OBSUV)--(IMDYN);
   
   \node[pname, text width=30cm] (J) at (26.5,-11.08) {COST FUNCTION: $J(\vec{u})=\|I_{tracer}- I_{FSLE}(\vec{u})\| + background\ term$};
   \draw[fleche] (IMTRA)--(J);
   \draw[fleche] (IMDYN)--(J);
    \end{tikzpicture} \\

\begin{center}
  \pbox{0.89\columnwidth}{}{linewidth=0.5mm,framearc=0.2,linecolor=black, fillstyle=gradient,gradangle=0,gradbegin=white,gradend=mycyan, gradmidpoint=0.0,framesep=3mm, shadowsize=0pt}{Goal: Find the optimal velocity $\vec{u}$ that minimizes the cost function J.} 
\end{center}
  }
    \vspace*{0.2cm}

    \begin{center}
      {\sc \bf \Large INVERSION METHOD \\[5mm]}
    \end{center}
    \vspace*{1.0cm}
    {\large
The strategy built up for the inversion of a tracer image is the following:
  
  $\bullet$ Compute a subspace of error to perturb the velocity \\
  $\bullet$ Explore the subspace of error \\
  $\bullet$  Use a simulated annealing algorithm to decrease the cost function: \\
%    \vspace*{0.2cm}
%\begin{center}
      \begin{minipage}{0.55\textwidth}
    \vspace*{1.0cm}
\pbox{0.99\columnwidth}{}{linewidth=0.5mm,framearc=0.2,linecolor=black, fillstyle=gradient,gradangle=0,gradbegin=white,gradend=white, gradmidpoint=0.0,framesep=3mm, shadowsize=0pt}{{\small
$\bullet$ Initialisation: \\
\indent  Background velocity $\vec{u}_b$: Velocity map derived from AVISO observation\\
\indent  Cost function: Distance between AVISO velocity and tracer observation\\
$\bullet$ Minimization: \\
\hspace{4em}Perturb the velocity: $\vec{u_p}=\vec{u} + \delta \vec{u}$ \\
\hspace{4em}Compute the cost function $J(\vec{u}_p)$ and the variation of the cost function $\delta J=J(\vec{u})-J(\vec{u}_p)$ \\
\hspace{4em}  \ding{226}  If $\delta J \leq 0$ $\rightarrow$ Accept perturbation ($\vec{u}=\vec{u}_p$) \\             
\hspace{4em}  \ding{226} If $\delta J > 0$ $\rightarrow$ Accept perturbation with probability $e^{-\frac{\delta J}{T}}$ \\
$\bullet$Exit when a minimum of the cost function is reached. }}
    \vspace*{1.0cm}
     \end{minipage}
%     \begin{center}
\hspace{0.5em}
     \begin{minipage}{0.41\textwidth}
    \vspace*{1.0cm}
        \pbox{0.99\columnwidth}{}{linewidth=0.5mm,framearc=0.2,linecolor=black, fillstyle=gradient,gradangle=0,gradbegin=white,gradend=white, gradmidpoint=0.0,framesep=3mm, shadowsize=0pt}{
          \vspace*{1.7cm}
          \myfig{./pict/s_atl/Jiter_20471_s_atl2.eps}{0.99}
          \small{Cost function as a function of the number of iteration (log-log plot) }
      }
    \vspace*{1.0cm} 
     \end{minipage}
%\end{center}
%\newline
%\newline
    %\vspace*{0.2cm}
\\ 
  $\bullet$ Use a Gibbs Sampler to assess possible solutions around the minimum of the cost function: As several perturbations result in similar values of the cost function, we compute a sample of all the potential solutions. \\ 



\begin{center}\small{(See \cite{Gaultier2012} for more details) }\end{center}
%    & \includegraphics[width=2.75cm]{./pict/aviso_h_020471_bin0.png}
%    & \includegraphics[width=2.35cm]{./pict/aviso_20471_s_atl2_oc.png}
%    & \includegraphics[width=2.75cm]{./pict/fsle_48_stat_reg_20471__bin0.png}
%    & \alt<2>{\includegraphics[width=2.73cm]{./pict/chl2006019.png}}
%      {\includegraphics[width=2.75cm]{./pict/sst_20471_small.png}}

   } %large
  } %pbox
\end{center}
 
%%%%%%%%%%%%%%%%%%%%%%%%%%%%%%%%%%%%%%%%%%%%%%%%%%%%%%%%%%%%%%%%%%%%%%%%%%%%%%%%%%%%%%%%%%%%%%%%%%%%%%%%%%%%%%%%%%%%%%%%%%%%%%%%%%%%%%%%%%%
  \begin{center}
  \pbox{0.90\columnwidth}{15cm}{linewidth=2mm,framearc=0.1,linecolor=mycyan, fillstyle=gradient,gradangle=0,gradbegin=white,gradend=white,gradmidpoint=1.0,framesep=1em}{
    \begin{center}
      {\sc \bf \Large Finite Size Lyapunov Exponents (FSLE)\\[5mm]}
    \end{center}

   %\vspace*{0.2cm}
\begin{minipage}{0.35\textwidth}
%\myfig{./pict/GND100L_ly_inst_temp_0529.eps}{0.9}
\myfig{./pict/EEL-R2BIG_FSLECHL_instat20_19690429.eps}{0.99}
\small{FSLE maximum lines plotted over Chlorophyll, data provided by a high resolution model}
\end{minipage}
    {\large
 \begin{minipage}{0.64\textwidth}
   \vspace*{0.2cm}
\ding{226} FLSE is the exponential rate at which two particles seperate from a distance $\delta _0$ to $\delta _f$: $ \lambda=\frac{1}{\tau}\ln{\frac{\delta_f}{\delta_0}}$.\\ 
It is a connection between sub-mesoscale dynamics and tracer stirring. \\ 
\ding{226}There are similar patterns between the maximum lines of FSLE (computed backward in time) and the tracer frontal structure (the norm of the gradient) \\
\ding{226}To compare the dynamics and the tracer image, the FSLE field and the normalized SST gradients are binarized: 
        $\hat{\lambda}=
   \left \{
   \begin{array}{r r r}
      0  &  if & \lambda<\lambda^s \\
      1  &  if & \lambda \ge \lambda^s \\
   \end{array}
   \right .
$
\end{minipage}
  } %large
  } %pbox
  \end{center}


%%%%%%%%%%%%%%%%%%%%%%%%%%%%%%%%%%%%%%%%%%%%%%%%%%%%%%%%%%%%%%%%%%%%%%%%%%%%%%%%%%%%%%%%%%%%%%%%%%%%%%%%%%%%%%%%%%%%%%%%%%%%%%%%%%%%%%%%%%%
  \begin{center}
%% \begin{pcolumn}{0.32}
  \pbox{0.90\columnwidth}{15cm}{linewidth=2mm,framearc=0.1,linecolor=mycyan, fillstyle=gradient,gradangle=0,gradbegin=white,gradend=white,gradmidpoint=1.0,framesep=1em}{
    \begin{center}
      {\sc \bf \Large TEST CASE\\[5mm]} 
    \end{center} 
 
    \vspace*{0.2cm}
    {\large  
    \vspace*{0.2cm}
\begin{minipage}{0.35\textwidth}
 
%    \begin{figure}
      \myfig{./pict/s_atl/s_atlb.eps}{0.99}
      \small{Chorophyll, in the South Atlantic Ocean. The study area is inside the pink frame}
%      \caption{South Atlantic Ocean}
%     \end{figure}
\end{minipage}
\begin{minipage}{0.65\textwidth}
  \begin{itemize}
    \item Time Range to build a subspace of error: 1998-2009, 595 velocity maps from AVISO (altimetric data)
    \item Background Velocity: AVISO map on 01/18/2006
    \item Velocity resolution: $1/3^o$, grid points : 13*17
    \item FSLE Resolution: $1/48^o$, grid points : 99*130
    \item Tracer field: SST or Chlorophyll data (MODIS sensor, L2 product) on 01/19/2003
    \item Resolution needed to detect filaments: $1/100^o$
    %resolution needed to detect filament $1/100^o$, to match fsle $1/50^o$
  \end{itemize}
\end{minipage}
%   \begin{minipage}{0.49\linewidth} 
%      \begin{pcolumn}{0.49}
%\ 
%AVISO ALTIMETRIC DATA AND MODIS SENSOR
%      \end{pcolumn}
%      \begin{pcolumn}{0.49}
% MODEL \\ 
%\ 
%SIMULATION IDEALISE HAUTE RESOLUTION COUPLEE PHYSIQUE BIOLOGIE\\
%
%    \end{pcolumn}
%   \end{minipage}
  } %large
  } %pbox
  \end{center}
 
%%%%%%%%%%%%%%%%%%%%%%%%%%%%%%%%%%%%%%%%%%%%%%%%%%%%%%%%%%%%%%%%%%%%%%%%%%%%%%%%%%%%%%%%%%%%%%%%%%%%%%%%%%%%%%%%%%%%%%%%%%%%%%%%%%%%%%%%%%%
\begin{center}
  \pbox{0.90\columnwidth}{37cm}{linewidth=2mm, framearc=0.1, linecolor=mycyan, fillstyle=gradient, gradangle=0, gradbegin=white, gradend=white, gradmidpoint=1.0, framesep=1em}{
    \begin{center}
      {\sc \bf \Large RESULTS\\[5mm]}
    \end{center}
    \vspace*{0.2cm}
    {\large
    \begin{tabular}{p{0.2em}cccl}
    %\hline
    & \small{SSH}
    & \small{Velocity field}
    & \small{FSLE}
    &\small{Chlorophyll} \\
    %\hline
    \rotatebox{90}{hspace{1cm}\small{OBSERVATION}}
    & \includegraphics[width=7.99cm]{./pict/s_atl/aviso_h_020471_bin0.eps}
    & \includegraphics[width=7.35cm]{./pict/s_atl/aviso_20471_s_atl2_oc.eps}
    & \includegraphics[width=7.99cm]{./pict/s_atl/fsle_48_stat_reg_20471__bin0.eps}
    & \includegraphics[width=7.93cm]{./pict/s_atl/chl2006019.eps} \\
  %   \hline
    % \hline
    \rotatebox{90}{\hspace{1cm}\small{CORRECTION}}
    & \includegraphics[width=7.9cm]{./pict/s_atl/aviso_h_020471_eof_bin1_oc.eps}
   & \includegraphics[width=7.3cm]{./pict/s_atl/aviso_20471_s_atl2_sa_oc.eps}
   & \includegraphics[width=7.95cm]{./pict/s_atl/fsle_48_stat_reg_20471__sa_bin1_oc.eps}
   & \\
 %\hline
  \end{tabular}
\\
The inversion method aims at correcting velocity vectors so as to follow tracer frontal structures.\\
\ding{226} 
Looking at the Chlorophyll image and the FSLE computed using the observed AVISO velocity, we can clearly see that the eddy is shifted in AVISO data.  \\
Looking at the corrected velocity and the corresponding FSLE, we can see that the eddy is well corrected. The corrected velocity is a better match to the tracer frontal structure than the AVISO derived velocity. 
\\
\vspace{1.0cm}
\ding{226}
{\bf Validation of the method using a model:} \\
This method has been tested on an idealized physico-biogeochemical model. 
Knowing the truth, the tracers SST and Chlorophyll are inverted to correct a perturbed velocity. The correction brought by the algorithm is assessed. Nearly 50\% of the error on the pertubation file is corrected. % how much the improvement brought by this algorithm
%Orrection of 50\%  
  } %large
  } %pbax 
  \end{center}


%%%%%%%%%%%%%%%%%%%%%%%%%%%%%%%%%%%%%%%%%%%%%%%%%%%%%%%%%%%%%%%%%%%%%%%%%%%%%%%%%%%%%%%%%%%%%%%%%%%%%%%%%%%%%%%%%%%%%%%%%%%%%%%%%%%%%%%%%%%%%
\begin{center}
  \pbox{0.90\columnwidth}{16.0cm}{linewidth=2mm,framearc=0.1,linecolor=mycyan,fillstyle=gradient,gradangle=0,gradbegin=white,gradend=white,gradmidpoint=1.0,framesep=1em}{
    \begin{center}{\sc \bf \Large  CONCLUSION\\[5mm]} \end{center}
    {\large
     \begin{center}
      \pbox{0.90\columnwidth}{}{linewidth=0.5mm,framearc=0.2,linecolor=black, fillstyle=gradient,gradangle=0,gradbegin=white,gradend=mycyan, gradmidpoint=0.0,framesep=3mm, shadowsize=0pt}{ We succeeded in correcting an altimetric mesoscale velocity field using a sub-mesoscale tracer observation from space }
      \end{center}
      \ding{226} The corrected velocity is more consistent with the tracer field than the background AVISO velocity derived from altimetry alone, and the uncertainty on this result is small. 
       However, the method still needs to be improved, since some areas of the velocity field are not accurately corrected. \\    
      \ding{226} We used a high resolution model with biogeochemics and physics coupled to refine the method. 
      Knowing the true sub-mesoscale velocity,  nearly 50\% of the error on the velocity can be corrected during the tracer inversion process. \\ %assessed how much the corrected velocity is a better match to the tracer than the mesoscale `observed' velocity. \\ 
      \ding{226} This study opens the way for the use of very high resolution altimeter data (in the context of SWOT and SARAL projects). The strategy proposed in here enables us to handle huge amount of data in models. 
%%PROSPECTS: Full DA on a physico biogeochemical model \\
%%                 Learn to deal with HR image and huge data set to think ahead SWOT and SARAL missions. \\ 
   
    } %large
  }  %pbox
 \end{center}
\end{multicols}
%%\end{minipage}

%%%%%%%%%%%%%%%%%%%%%%%%%%%%%%%%%%%%%%%%%%%%%%%%%%%%%%%%%%%%%%%%%%%%%%%%%%%%%%%%%%%%%%%%%%%%%%%%%%%%%%%%%%%%%%%%%%%%%%%%%%%%%%%%%%%%%%%%%%%%
%\begin{center}
%\pbox{0.96\columnwidth}{34cm}{linewidth=2mm,framearc=0.1,linecolor=mycyan, fillstyle=gradient,gradangle=0,gradbegin=white,gradend=white,gradmidpoint=1.0,framesep=1em}{
%  \begin{center}
%    {\sc \bf \Large METHOD \\[5mm]}
%  \end{center} 
%  {\large
       
%    \begin{pcolumn}{0.32}
%      \vspace{1cm}
%      \begin{center}{\bf Finite-Size Lyapunov Exponents (FSLE)} \end{center}
    
%      \ding{52} FSLE measure stirring in a fluid, It is a connection between
%sub-mesoscale dynamics and biologic stirring.\\ 
%      \ding{52} FSLE is the exponential rate at which two particles separate from a distance $\delta_0$ to $\delta_f$:
%%\begin{equation*}
%      $ \lambda=\frac{1}{\tau}\ln{\frac{\delta_f}{\delta_0}}$. \\ 
%% \label{lyap}
%%\end{equation*}
%%      The particles are advected backward in time so that maximum lines of FSLE represent the unstable manifolds.
%      Technically, four particles are initially at a distance $\delta_0$ from each other, we compute the time $\tau$ for these particles to reach the distance $\delta_f$ from the other four particles. 
%      \vspace{1em}

%      \begin{minipage}{0.49\textwidth}
%      \pbox{0.9\columnwidth}{}{linewidth=0.5mm,framearc=0.2,linecolor=black, fillstyle=gradient,gradangle=0,gradbegin=white,gradend=white, gradmidpoint=0.0,framesep=3mm, shadowsize=0pt}{
%        \myfig{./pict/fsle_48_stat_reg_19904_med.eps}{0.8}
%        \small{FSLE (in day$^{-1}$) derived from the geostrophic AVISO velocity June 30$^{th}$, 2004} 
%      }
%      \end{minipage}
%      \begin{minipage}{0.49\textwidth}
%        There are similar patterns between the maximum lines of Lyapunov exponents and SST frontal structure
%(that is to say the norm of the gradient). 
%      \end{minipage}

%      \vspace{1em}

%      \vfill
%      \ding{52} We want to find the FSLE field as close as possible to the normalized tracer gradient. Therefore, the FSLE field and the normalized SST gradients are binarized in order to compare those two physically different variables. 
%        $$\hat{\lambda}=
%   \left \{
%   \begin{array}{r r r}
%      0  &  if & \lambda<\lambda^s \\
%      1  &  if & \lambda \ge \lambda^s \\
%   \end{array}
%   \right .$$
%    \end{pcolumn}            
%    \vline 
%    \begin{pcolumn}{0.32}
%      \vspace{1cm}
%      \begin{center}{\bf The Cost Function} \end{center}
%      \ding{52}The cost function $J$ measures the distance between the binarized normalized gradient of SST $\hat{\lambda}_o$ and the binarized FSLE $\hat{\lambda}({\bf u})$ (as a proxy of the velocity ${\bf u}$). 
%A background term is added so that the corrections applied on the velocity are small.
%%It also takes into account the fact that the corrected velocity should not be too far from the background velocity
%      $$J({\bf u}) = \mu \; || \hat{\lambda}_o - \hat{\lambda}({\bf u}) ||^2+\ background \ term$$
%%             +  ( {\bf u} - {\bf u}_b )^T  {\bf B}^{-1}
%%                ( {\bf u} - {\bf u}_b )$$ 
%       The minimum of the cost function corresponds to the velocity that is the most consistent with the tracer.

%      \vspace{1em}
%      \begin{center}
%      \pbox{0.9\columnwidth}{}{linewidth=0.5mm,framearc=0.2,linecolor=black, fillstyle=gradient,gradangle=0,gradbegin=white,gradend=white, gradmidpoint=0.0,framesep=3mm, shadowsize=0pt}{
%        \begin{minipage}{0.45\textwidth}
%          \myfig{./pict/binsst_bin0.eps}{0.82}
%          \small{Binarization of the norm of the SST gradient on July 2$^{nd}$, 2004}
%        \end{minipage}
%        \hfill
%        \begin{minipage}{0.47\textwidth}
%          \myfig{./pict/pfsle_48_stat_reg_19904_med.eps}{0.8}
%          \small{Binarization of the FSLE derived from the AVISO velocity on June 30$^{th}$, 2004}
%        \end{minipage}
%      }
%      \end{center}
%      \vspace{1em}

%      \ding{52} We create a velocity panel using Principal Component Analysis with an ensemble of velocities. \\ 
%      Let S = $|u^{(1)}, u^{(2)}, ..., u^{(r)}|$ be the first r EOFs. 
%      We consider velocity errors with zero mean and covariance of the ensemble of velocities $S$: $\delta u \sim \mathcal{N}(0,SS^T)$ 
%so that a perturbed velocity is $\textbf{u}_k = \bar{\textbf{u}} + \sum_{i=0}^r{a_k^{(i)}\textbf{u}^{(i)}}$, with $a_k$ a perturbation vector. 
%%     We explore the subspace of errors to find the most suitable correction.  
%%     $$\textbf{u}_k = \bar{\textbf{u}} + \sum_{i=0}^n{\underbrace{a_k^i}_{Eigenvalue}\underbrace{\textbf{u}^i}_{EOF_{}}}$$
%%Find the velocity that minimizes the cost function among the perturbed velocities
%      \hfill
%    \end{pcolumn}
%    \vline
%    \begin{pcolumn}{0.32}
%      \vspace{1cm}
%      \begin{center}{\bf Technical issues to find the optimal solution} \end{center}
%      \ding{52} We aim at decreasing the cost function, exploring the sub-space of velocity errors previously defined. 
%      Nevertheless, the cost function $J(u)$ is quite irregular and many local minima can be found. 
%      To avoid being stuck in one of them, we decrease the cost function using a Simulated Annealing. \\
%%      At each step, a perturbation is computed and the resulting cost function is calculated, if it is lower than the previous one, we adopt the perturbation, otherwise, we reject the perturbation with a certain probability. \\ 
      
%      \vspace{1em}

%      \begin{minipage}{0.49\textwidth}
%        \pbox{0.9\columnwidth}{}{linewidth=0.5mm,framearc=0.2,linecolor=black, fillstyle=gradient,gradangle=0,gradbegin=white,gradend=white, gradmidpoint=0.0,framesep=3mm, shadowsize=0pt}{
%          \myfig{./pict/Jiter_real_19904_obs0_22_bin0.eps}{0.8}
%          \small{Cost function as a function of the number of iteration (log-log plot) }
%      }
%      \end{minipage}
%      \begin{minipage}{0.49\textwidth}
%        At each step, a perturbation is computed and the resulting cost function is calculated, if it is lower than the previous one, we adopt the perturbation, otherwise, we reject the perturbation with a certain probability. \\
%      \end{minipage}

%      \vspace{1em}

%      \ding{52} As several perturbations result in similar values of the cost function, we need to find a strategy to compute accurately the final result. 
%      To do so, we use a Gibbs Sampler. 
%      In other words, we assess all the perturbations that are potential solutions to the problem. 
%      The mean of all the potential solution gives us the optimal solution, and the variance indicates how reliable this optimal solution is. 
       
%      \vfill
%    \end{pcolumn}
%  } %large
%} %pbox
%\end{center}
 
%\vspace*{0.2cm}
 
%\begin{multicols}{2}
 
 
%%BIBLIO
{\small
   \bibliographystyle{./bst/jpo}
 %\bibliographystyle{alpha}
  \bibliography{../biblio/biblio_th}
}
  
\begin{center}
  {\bf Laboratoire des Ecoulements G\'eophysiques et Industriels, 
1023 rue de la piscine, Domaine Universitaire, 38400 Saint Martin d'Heres, FRANCE 
e-mail: lucile.gaultier@hmg.inpg.fr}
\end{center}
 
\end{poster}
\end{document}
