%\documentclass[portrait,a0b,final,a4resizeable]{a0poster}
\documentclass[portrait,a0b,final]{a0poster}
%%% Option "a4resizeable" enables the poster to be resizable using the command: psresize -pa4 poster.ps poster-a4.ps
%   with landscape option: psresize -W800 -H600 poster.ps poter-a4.ps
%   and replace in "poster-a4.ps" the line:  
%   %%BoundingBox: 0 0 2594 3402  with  %%BoundingBox: 0 0 600 800 
%   For final printing, please remove option "a4resizeable" !!
\usepackage{epsfig}
\usepackage{multicol}
\usepackage{natbib}
\usepackage{pstricks,pst-grad}
\usepackage{graphicx}
\usepackage{amsbsy,amsmath}
\usepackage{tikz} %add schemes
\usetikzlibrary{shapes} %add diamonds shape to schemes

%%%%%%%%%%%%%%%%%%%%%%%%%%%%%%%%%%%%%%%%%%%%
% Definition of some variables and colors %%
%%%%%%%%%%%%%%%%%%%%%%%%%%%%%%%%%%%%%%%%%%%%

\newcommand{\unnumberedcaption}%

\newcommand{\BIBand}{and}

\newcommand{\Deg}{$^{\circ} ~$}
%\DeclareTextSymbol{\deg}{T1}{6}
%\DeclareTextSymbol{\degre}{OT1}{23}

\setlength{\bibsep}{0.0pt}
\def\refname{\normalsize \bf References}
%%\renewcommand{\rho}{\varrho}
%%\renewcommand{\phi}{\varphi}
%\setlength{\columnsep}{3cm}
%\setlength{\columnseprule}{2mm}
%\setlength{\parindent}{0.0cm}

\renewcommand{\vec}{\underline}
%\renewcommand{\rho}{\varrho}
%\renewcommand{\phi}{\varphi}
\setlength{\columnsep}{0.1cm}
\setlength{\columnseprule}{0mm}
\setlength{\parindent}{0.0cm}

%%%%%%%%%%%%%%%%%%%%%%%%%%%%%%%%%%%%%%%%%%%%%%%%%%%%
%%%               Background                     %%%
%%%%%%%%%%%%%%%%%%%%%%%%%%%%%%%%%%%%%%%%%%%%%%%%%%%%

\newcommand{\background}[3]{
  \newrgbcolor{cgradbegin}{#1}
  \newrgbcolor{cgradend}{#2}
  \psframe[fillstyle=gradient,gradend=cgradend,
  gradbegin=cgradbegin,gradmidpoint=#3](0.,0.)(1.\textwidth,-1.\textheight)
}


%%%%%%%%%%%%%%%%%%%%%%%%%%%%%%%%%%%%%%%%%%%%%%%%%%%%
%%%                Poster                        %%%
%%%%%%%%%%%%%%%%%%%%%%%%%%%%%%%%%%%%%%%%%%%%%%%%%%%%

\newenvironment{poster}{
  \begin{center}
  \begin{minipage}[c]{1.0\textwidth}
}{
  \end{minipage} 
  \end{center}
}


%%%%%%%%%%%%%%%%%%%%%%%%%%%%%%%%%%%%%%%%%%%%%%%%%%%%
%%%                pcolumn                       %%%
%%%%%%%%%%%%%%%%%%%%%%%%%%%%%%%%%%%%%%%%%%%%%%%%%%%%

\newenvironment{pcolumn}[1]{
  \begin{minipage}{#1\textwidth}
%  \begin{center}
}{
%  \end{center}
  \end{minipage}
}

%%%%%%%%%%%%%%%%%%%%%%%%%%%%%%%%%%%%%%%%%%%%%%%%%%%%
%%%                pboxx                         %%%
%%%%%%%%%%%%%%%%%%%%%%%%%%%%%%%%%%%%%%%%%%%%%%%%%%%%
\newrgbcolor{lcolor}{0. 0. 0.80}
\newrgbcolor{gcolor1}{1. 1. 1.}
\newrgbcolor{gcolor2}{.80 .80 1.}

\newcommand{\pboxx}[4]{
\begin{minipage}[t][#2][t]{#1}
#4
\end{minipage}
}

%%%%%%%%%%%%%%%%%%%%%%%%%%%%%%%%%%%%%%%%%%%%%%%%%%%%
%%%                pbox                          %%%
%%%%%%%%%%%%%%%%%%%%%%%%%%%%%%%%%%%%%%%%%%%%%%%%%%%%
\newrgbcolor{lcolor}{0. 0. 0.80}
\newrgbcolor{gcolor1}{1. 1. 1.}
\newrgbcolor{gcolor2}{.80 .80 1.}

\newcommand{\pbox}[4]{
         \psshadowbox[#3]{
            \begin{minipage}[t][#2][t]{#1} #4
            \end{minipage}
         }
}


%%%%%%%%%%%%%%%%%%%%%%%%%%%%%%%%%%%%%%%%%%%%%%%%%%%%
%%%                pfig                          %%%
%%%%%%%%%%%%%%%%%%%%%%%%%%%%%%%%%%%%%%%%%%%%%%%%%%%%
% \pfig - replacement for \figure
% necessary, since in multicol-environment
% \figure won't work

\newcommand{\pfig}[3][0]{
\begin{center}
  \vspace{1cm}
  \includegraphics[width=#3\hsize,angle=#1]{#2}
  \nobreak\medskip
\end{center}}


%%%%%%%%%%%%%%%%%%%%%%%%%%%%%%%%%%%%%%%%%%%%%%%%%%%%
%%%                pcaption                      %%%
%%%%%%%%%%%%%%%%%%%%%%%%%%%%%%%%%%%%%%%%%%%%%%%%%%%%
% \pcaption - replacement for \caption
% necessary, since in multicol-environment \figure and
% therefore \caption won't work

%\newcounter{figure}
\setcounter{figure}{1}
\newcommand{\pcaption}[1]{
  \vspace{0.cm}
  \begin{quote}
    {{\sc Figure} \arabic{figure}: #1}
  \end{quote}
  \vspace{1cm}
  \stepcounter{figure}
}
%%%%%%%%%%%%%%%%%%%%%%%%%%%%%%%%%%%%%%%%%%%%%%%%%%%%
%%%                myfig                         %%%
%%%%%%%%%%%%%%%%%%%%%%%%%%%%%%%%%%%%%%%%%%%%%%%%%%%%
% \myfig - replacement for \figure
% necessary, since in multicol-environment 
% \figure won't work

\newcommand{\myfig}[3][0]{
\begin{center}
  \vspace{0.10cm}
  \includegraphics[width=#3\hsize,angle=#1]{#2}
%  \nobreak\medskip
\end{center}}


%%%%%%%%%%%%%%%%%%%%%%%%%%%%%%%%%%%%%%%%%%%%%%%%%%%%
%%%                mycaption                     %%%
%%%%%%%%%%%%%%%%%%%%%%%%%%%%%%%%%%%%%%%%%%%%%%%%%%%%
% \mycaption - replacement for \caption
% necessary, since in multicol-environment \figure and
% therefore \caption won't work

%\newcounter{figure}
\setcounter{figure}{1}
\newcommand{\mycaption}[1]{
  \vspace{0.0cm}
  \begin{quote}
    {\small{#1}}
  \end{quote}
  \vspace{0cm}
  \stepcounter{figure}
}


% MES DEFINITIONS DES COULEURS:
\newrgbcolor{lcolor}{0. 0. 0.80}
\newrgbcolor{gcolor1}{1. 1. 1.}
\newrgbcolor{gcolor2}{.80 .80 1.}
\newrgbcolor{gradbegin}{0 0 0.8}
\newrgbcolor{gradend}{1 1 1}
\newrgbcolor{lightblue}{0. 0. 0.80}
\newrgbcolor{lightyellow}{1 1 0.}
\newrgbcolor{customcolor}{1 0.75 0.}
\newrgbcolor{customcolor2}{1 0.5 0.}
\newrgbcolor{customcolor3}{0 0.8 0.2}
\newrgbcolor{white}{1. 1. 1.}
\newrgbcolor{whiteblue}{.80 .80 1.}
\newrgbcolor{mycyan}{.20 .60 .60}

%%%%%%%%%%%%%%%%%%%%%%%%%%%%%%%%%%%%%%%%%%%%%%%%%%%%%%%%%%%%%%%%%%%%%%
%%% Begin of Document
%%%%%%%%%%%%%%%%%%%%%%%%%%%%%%%%%%%%%%%%%%%%%%%%%%%%%%%%%%%%%%%%%%%%%%

\begin{document}
\background{0.2 0.6 0.6}{1. 1. 1.}{0.5}

\vspace*{2cm}

%\newrgbcolor{lightblue}{0. 0. 0.80}
%\newrgbcolor{white}{1. 1. 1.}
%\newrgbcolor{whiteblue}{.80 .80 1.}

%\newrgbcolor{darkred}{0.6 0 0}
%\newrgbcolor{lightblue2}{0.2 0.2 1}
%\newrgbcolor{lightblue}{0.5 0.5 1}
%\newrgbcolor{white}{1. 1. 1.}
%\newrgbcolor{whiteblue}{.80 .80 1.}
%\newrgbcolor{mycyan}{rgb}{0.2,0.6,0.6}



\begin{poster}
%%%%%%%%%%%%%%%%%%%%%
%%% Header
%%%%%%%%%%%%%%%%%%%%%
 \begin{center}
 \begin{pcolumn}{0.99}
 \begin{center}
   \pbox{0.95\textwidth}{}{linewidth=2mm,framearc=0.3,linecolor=mycyan,shadowcolor=white,fillstyle=gradient,gradangle=0,gradbegin=white,gradend=white,gradmidpoint=1.0,framesep=1em}{
 
%%% Logo LPO
  \begin{minipage}[l][9cm][c]{0.1\textwidth}
  \begin{center}
    \includegraphics[width=0.85\textwidth,angle=0]{./pict/logo_cnrs.eps}
    \includegraphics[width=0.85\textwidth,angle=0]{./pict/logo_cnes2.eps}
  \end{center}
  \end{minipage}
%%% Titre
  \begin{minipage}[c][9cm][c]{0.78\textwidth}
  \begin{center}
     {\sc \huge \bf On the inversion of sub-mesoscale information to correct mesoscale velocity}\\[10mm]
     {\Large Lucile Gaultier$^{1}$, Jacques Verron$^{1}$, Jean-Michel Brankart$^{1}$, Olivier Titaud$^{1}$ and Pierre Brasseur$^{1}$} \\[5mm]
     {\large {$^{1}$ \it Laboratoire des Ecoulements G\'eophysiques et Industriels, UMR 5519, CNRS and Universit\'e de Grenoble, Grenoble, France} %\\
%     %{$^{2}$ \it Departement of Oceanography, The Florida State University, Tallahassee, USA}\\
     }
 
  \end{center}
  \end{minipage}
%%% Logo-UBO
  \begin{minipage}[r][9cm][c]{0.1\textwidth}
  \begin{center}
    \includegraphics[width=0.8\textwidth,angle=0]{./pict/logo_meom.eps}
    \includegraphics[width=0.8\textwidth,angle=0]{./pict/logo_legi.eps}
  \end{center}
  \end{minipage}
 
  }  %pbox
\end{center}
\end{pcolumn}
\end{center}
 
 
%\vspace*{2cm}
 
%%% Begin of Multicols-Environment
\begin{multicols}{2}
 
 \newcommand{\etal}{{\it et al.}}
 \newcommand{\DegN}{$^{\circ}$N~}
 \newcommand{\DegW}{$^{\circ}$W~}
 \newcommand{\DegE}{$^{\circ}$E~}
 \newcommand{\DegS}{$^{\circ}$S~}
 
% %%%%%%%%%%%%%%%%%%%%%%%%%%%%%%%%%%%%%%%%%%%%%%%%%%%%%%%%%%%%%%%%%%%%%%%%%%%%%%%%%%%%%%%%%%%%%%%%%%%%%%%%%%%%%%%%%%%%%%%%%%%%%%%%%%%%%%%%%%%%
% %%%% Introduction
 \begin{center}
% \begin{pcolumn}{0.32}
   \pbox{0.92\columnwidth}{30cm}{linewidth=2mm,framearc=0.1,linecolor=mycyan, fillstyle=gradient,gradangle=0,gradbegin=white,gradend=white, gradmidpoint=1.0,framesep=1em}{
   \begin{center} 
     { \bf \Large CONTEXT \\[5mm]}%MOTIVATIONS 
   \end{center}
   {\large
 
   This poster reports on the correction of mesoscale altimetric observation using sub-mesoscale tracer information. 
   Altimetric satellites provides one map a week at a mesoscale resolution (more than 1/8\Deg) whereas tracer sensors have a fine resolution in space (as low as few hundred meters) and time (a map a week).
   Tracer observation can compensate for the lack of resolution in space and time of Altimetric data, if we can extract dynamic information from the tracer image. 
   Some studies brought to light the connection between mesoscale velocities and tracer patterns% \citep{Dovidio2004, Lehahn2007}, 
but correcting mesoscale velocity using tracers has never been done before. We present in here the strategy we made up to correct mesoscale altimetric velocity using a tracer image.
   \begin{center}
   \begin{tikzpicture}
    \node[color=blue, text width=10cm, text centered] (UV) at (16,6.8) {Mesoscale field};
    \node[color=black, text width=09.73cm, text centered] (pUV) at (16,0.48) {
        \includegraphics[width=1\linewidth]{pict/aviso_20079_tas.eps}};
    \node[color=black, text width=15cm, text centered] (cUV) at (16,-7) {\small{Velocity map, Tasmania region, December 22, 2004}};
%        \mycaption{Velocity map, Tasmania region, December 22, 2004}
%         };
    \node[color=green, text width=15cm, text centered] (TRA) at (0,6.7) {Sub-mesoscale tracer image};
    \node[color=black, text width=10cm, text centered] (pTRA) at (0,0) {
        \includegraphics[width=1\linewidth]{pict/A2004358041000_L2_LAC_OC.eps}};
    \node[color=black, text width=15cm, text centered] (cUV) at (0,-7) {\small{Chlorophyll, Tasmania region, December 22, 2004}};
%        \mycaption{Chlorophyll, Tasmania region, December 22, 2004}
 %      };
    \node[draw] (int) at (8,2.1) {\large{?}};
%  \draw[arrows={triangle 45-triangle 45}] (pTRA)--(16,0);
  \draw[->] (pTRA)--(10,0);
  \end{tikzpicture}\\ 
$\Rightarrow$Use of FSLE as a go-between variable to enable the velocity field and the tracer image to discuss together

  \end{center}
   } %large
   } %pbox
% gend{pcolumn}
 \end{center}
 
 \vspace*{0.1cm}
 
%%%%%%%%%%%%%%%%%%%%%%%%%%%%%%%%%%%%%%%%%%%%%%%%%%%%%%%%%%%%%%%%%%%%%%%%%%%%%%%%%%%%%%%%%%%%%%%%%%%%%%%%%%%%%%%%%%%%%%%%%%%%%%%%%%%%%%%%%%%
 \begin{center}
% \begin{pcolumn}{0.32}
   \pbox{0.92\columnwidth}{30cm}{linewidth=2mm,framearc=0.1,linecolor=mycyan, fillstyle=gradient,gradangle=0,gradbegin=white,gradend=white,gradmidpoint=1.0,framesep=1em}{
   \begin{center}
     {\sc \bf \Large TEST CASE\\[5mm]} 
   \end{center} 
 
   \vspace*{0.2cm}
 
   {\large
    \fbox{
  \begin{minipage}{0.49\linewidth}
     \myfig[0]{./pict/aviso_h_019547_eof.eps}{0.8}
     \small{SSH (in cm) in the Western Basin of the Mediterranean Sea. The test case is located in the purple rectangle}
   \end{minipage}
    }
   \begin{minipage}{0.49\linewidth}
 The chosen area is located in the Western basin of the Mediterranean Sea (from 38.2\DegN to 40\DegN and from 4.8\DegE to 8\DegE).
It has been chosen because of the strong filament signature. The study date, July 2$^{nd}$, 2004 is selected so that there is no cloud on our area and also because of the presence of a neat gyre on the SST image. %The date of the velocity is June 30$^{th}$, 2004. It is chosen to be the available AVISO velocity field as close as possible to the study day.
   \end{minipage}
   \vfill
%\vspace{0.8cm}
Two satellite observation data sets are considered: \\ 
\vfill 
%    \vspace{8mm}
    \begin{tabular}{|p{20cm}|p{20cm}|}
      \hline
       High resolution SST images from MODIS sensor at a resolution of 1 km: & 
       Maps of velocity derived from AVISO altimetric observations at a resolution of 1/8\Deg: \\ 
      \hline
       \myfig[0]{./pict/A2004184123500_L2_LAC_SST.eps}{0.6} 
       \small{SST observation from MODIS sensor (in \Deg C) on July 2$^{nd}$, 2004} &
       \myfig[0]{./pict/aviso_velmap19904_med.eps}{0.6} 
       \small{Geostrophic Velocity (in m.s$^{-1}$) over the SSH (in cm) on the day June 30$^{th}$, 2004 from AVISO mapped products} \\ 
       \hline
%       \mycaption{SST observation from MODIS sensor (in \Deg C) on July 2$^{nd}$, 2004} &
%       \mycaption{Geostrophic Velocity (in m.s$^{-1}$) over the SSH (in cm) on the day June 30$^{th}$, 2004 from AVISO mapped products} \\ 
%        \hline
     \end{tabular}
%The direct measure of the distance between \textbf{u} and tracer is not possible
%$\rightarrow$ We use Finite-Size Lyapunov Exponents (FSLE) as a go-between variable

%\begin{tabular}{cc}
%     1 &e \\
%     2 & e \\
%\eend{tabular}

%    \vline
 %   \begin{minipage}{0.49\linewidth}
%       Maps of velocity derived from AVISO altimetric observations at a resolution of 1/8\Deg.
%       \fbox{
%        \myfig[0]{./pict/aviso_velmap19904_med.eps}{0.6}
%        \mycaption{Geostrophic Velocity (in m.s$^{-1}$) over the SSH (in cm) on the day June 30$^{th}$, 2004 from AVISO mapped products}
%        }
%     \end{minipage}

%\begin{enumerate}
%\item High resolution Sea Surface Temperature (SST) images from MODIS sensor at a resolution of 1 km.
%\item Maps of velocity derived from AVISO altimetric observations at a resolution of 1/8\Deg.
%\end{enumerate}
%   \fbox{
%  \begin{minipage}{0.49\linewidth}
%     \myfig[0]{./pict/aviso_h_019547_eof.eps}{0.8}
    % \mycaption{SSH (in cm) in the Western Basin of the Mediterranean Sea. The test case is located in the purple rectangle}
%   \end{minipage}
%    }
%   \begin{minipage}{0.49\linewidth}
% The chosen area is located in the Western basin of the Mediterranean Sea (from 38.2\DegN to 40\DegN and from 4.8\DegE to 8\DegE).
%It has been chosen because of the strong filament signature. The study date, July 2$^{nd}$, 2004 is selected so that there is no cloud on our area and also because of the presence of a neat gyre on the SST image. %The date of the velocity is June 30$^{th}$, 2004. It is chosen to be the available AVISO velocity field as close as possible to the study day.
%   \end{minipage}


 %\myfig[0]{./figures/2mld_compare.eps}{0.6}
 %\pcaption{Mixed Layer Depth in North Atlantic (march 2009 monthly mean) in ORCA025.L75-MJM91b (left) and ORCA025.L75-MJM95 (right)}       
  
         
 %\myfig[0]{./figures/tao_profiles.ORCA025.L75-MJM91b_ORCA025.L75-MJM95.eps}{0.6}
 %\pcaption{TAO profiles (year 2009 annual mean) in ORCA025.L75-MJM91b (red) and ORCA025.L75-MJM95 (black) compared to climatological data} 
 
 
   } %large
   } %pbox
% \end{pcolumn}
 \end{center}
 
%\vspace*{0.2cm}
\end{multicols}


%%%%%%%%%%%%%%%%%%%%%%%%%%%%%%%%%%%%%%%%%%%%%%%%%%%%%%%%%%%%%%%%%%%%%%%%%%%%%%%%%%%%%%%%%%%%%%%%%%%%%%%%%%%%%%%%%%%%%%%%%%%%%%%%%%%%%%%%%%%%
 \begin{center}
   \pbox{0.95\columnwidth}{30cm}{linewidth=2mm,framearc=0.1,linecolor=mycyan, fillstyle=gradient,gradangle=0,gradbegin=white,gradend=white,gradmidpoint=1.0,framesep=1em}{
   \begin{center}
     {\sc \bf \Large METHOD \\[5mm]}
   \end{center} 
   {\large
       
    \begin{pcolumn}{0.32}
    \begin{center}{\bf Finite-Size Lyapunov Exponents (FSLE)} \end{center}
    
     $\bullet$ FSLE measure stirring in a fluid, It is a connection between
sub-mesoscale dynamics and biologic stirring.\\ 
     $\bullet$ FSLE is the exponential rate at which two particles separate from a distance $\delta_0$ to $\delta_f$:
%\begin{equation*}
$$ \lambda=\frac{1}{\tau}\ln{\frac{\delta_f}{\delta_0}}$$
%% \label{lyap}
%\end{equation*}
      The particles are advected backward in time so that maximum lines of FSLE represent the unstable manifolds.
     \myfig{./pict/fsle_48_stat_reg_19904_med.eps}{0.6}
     \small{FSLE (in day$^{-1}$) derived from the geostrophic AVISO velocity on the day June 30$^{th}$, 2004} 

    \end{pcolumn}            
    \vline 
    \begin{pcolumn}{0.32}
    \begin{center}{\bf Maximum Probability Estimator} \end{center}
      Explore subspace of error\\ 
    Cost function 
$$J({\bf u}) = \mu \; || \hat{\lambda}_o - \hat{\lambda}({\bf u}) ||^2
             +  ( {\bf u} - {\bf u}_b )^T  {\bf B}^{-1}
                ( {\bf u} - {\bf u}_b )$$ 
    Simulated Annealing\\
    \end{pcolumn}
    \vline
    \begin{pcolumn}{0.32}
    \begin{center}{\bf Minimum Variance Estimator} \end{center}
     Gibbs'Sampling\\ 
     Rejection algo\\ 
    \end{pcolumn}
    

 %  \myfig[0]{./figures/glo_change_inS.eps}{0.33}
 %  \pcaption{Global change in salinity (compared to levitus) along depth in 2009 in MJM91b (black, no restoring at all), MJM95 (red),MAL951 (green), MAL952 (cyan), MAL953 (magenta)} 
  
 
   } %large
   } %pbox
 
 \end{center}
 
%\vspace*{0.2cm}
 
\begin{multicols}{2}
%%%%%%%%%%%%%%%%%%%%%%%%%%%%%%%%%%%%%%%%%%%%%%%%%%%%%%%%%%%%%%%%%%%%%%%%%%%%%%%%%%%%%%%%%%%%%%%%%%%%%%%%%%%%%%%%%%%%%%%%%%%%%%%%%%%%%%%%%%%%%
 \begin{center}
   \pbox{0.9\columnwidth}{30cm}{linewidth=2mm,framearc=0.1,linecolor=mycyan,fillstyle=gradient,gradangle=0,gradbegin=white,gradend=white,gradmidpoint=1.0,framesep=1em}{
   \begin{center}
     {\sc \bf \Large SENSITIVITY STUDY \\[5mm]}
   \end{center} 
   {\large
 
   \begin{center}
   \begin{minipage}{0.49\linewidth}
 
 %    \myfig[0]{./figures/logT_temp_bio.eps}{0.8}
 %    \pcaption{Global temperature drifts after 10 years of run in ORCA025-G84 (red) and ORCA025-G84NB (black)} 
 TTTTTTTTTTT
   \end{minipage}
   \hfill
   \begin{minipage}{0.49\linewidth}
 
 %  \myfig[0]{./figures/map_diff_temp_bio.eps}{0.9}
 %  \pcaption{Temperature difference between ORCA025-G84 and ORCA025-G84NB at 50 meters}   
 TTTTT
   \end{minipage}
   \end{center}
 
   } %Large
   } %pbox
 \end{center}
 
 %\vspace*{0.2cm}

 %%%%%%%%%%%%%%%%%%%%%%%%%%%%%%%%%%%%%%%%%%%%%%%%%%%%%%%%%%%%%%%%%%%%%%%%%%%%%%%%%%%%%%%%%%%%%%%%%%%%%%%%%%%%%%%%%%%%%%%%%%%%%%%%%%%%%%%%%%%%%%
 \begin{center}
   \pbox{0.9\columnwidth}{30cm}{linewidth=2mm,framearc=0.1,linecolor=mycyan,fillstyle=gradient,gradangle=0,gradbegin=white,gradend=white,gradmidpoint=1.0,framesep=1em}{
   \begin{center}{\sc \bf \Large  RESULTS} \end{center}
     {\large
 
 bla bla
 
 
 }
 }
 \end{center}

%%%%%%%%%%%%%%%%%%%%%%%%%%%%%%%%%%%%%%%%%%%%%%%%%%%%%%%%%%%%%%%%%%%%%%%%%%%%%%%%%%%%%%%%%%%%%%%%%%%%%%%%%%%%%%%%%%%%%%%%%%%%%%%%%%%%%%%%%%%%%%
 %%%%% Conclusion
 %\begin{center}
 %\pbox{.9\columnwidth}{}{linewidth=1mm,framearc=0,linecolor=white,
 %  fillstyle=gradient,gradangle=0,gradbegin=white,gradend=lightblue,
 %  gradmidpoint=1.0,framesep=1em}{
 %  \begin{center}{\sc \bf \Large  CONCLUSIONS} \end{center}
 %    {\large
 %
 %bla bla
 %
 %
 %}
 %}
 %\end{center}
 
 
%%BIBLIO
 {\tiny
 \bibliographystyle{./bst/jpo.bst}
 \bibliography{../biblio/biblio_th}
 }
 
 
\end{multicols}
\begin{center}
 
  {\bf Laboratoire des Ecoulements G\'eophysiques et Industriels, 
1023 rue de la piscine, Domaine Universitaire, 38400 Saint Martin d'Heres, FRANCE 
e-mail: lucile.gaultier@hmg.inpg.fr}
\end{center}
 
\end{poster}
\end{document}
